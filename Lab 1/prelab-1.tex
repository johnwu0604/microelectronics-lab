\documentclass[12pt]{article}
\usepackage[english]{babel}
\usepackage{natbib}
\usepackage{url}
\usepackage[utf8x]{inputenc}
\usepackage{amsmath}
\usepackage{graphicx}
\graphicspath{{images/}}
\usepackage{parskip}
\usepackage{fancyhdr}
\usepackage{vmargin}
\usepackage{subfig}
\usepackage{float}
\setmarginsrb{3 cm}{2.5 cm}{3 cm}{2.5 cm}{1 cm}{1.5 cm}{1 cm}{1.5 cm}

\title{2.1 - Differential Amplifier (Prelab)}                             % Title
\author{John Wu \\ Ian Smith \\ Bilal Mohammed}                               % Author
\date{\today}                                           % Date

\makeatletter
\let\thetitle\@title
\let\theauthor\@author
\let\thedate\@date
\makeatother

\pagestyle{fancy}
\fancyhf{}
\rhead{\theauthor}
\lhead{\thetitle}
\cfoot{\thepage}

\begin{document}

%%%%%%%%%%%%%%%%%%%%%%%%%%%%%%%%%%%%%%%%%%%%%%%%%%%%%%%%%%%%%%%%%%%%%%%%%%%%%%%%%%%%%%%%%

\begin{titlepage}
    \centering
    \vspace*{0.5 cm}
    \includegraphics[scale = 0.07]{mcgill-logo.png}\\[1.0 cm]   % University Logo
    \textsc{\LARGE McGill University}\\[1.0 cm]   % University Name
    \textsc{\Large ECSE 335}\\[0.5 cm]               % Course Code
    \textsc{\large Microelectronic Labs}\\[0.5 cm]               % Course Name
    \rule{\linewidth}{0.2 mm} \\[0.4 cm]
    { \huge \bfseries \thetitle}\\
    \rule{\linewidth}{0.2 mm} \\[1.5 cm]
    \begin{minipage}{0.4\textwidth}
        \begin{flushleft} \large
            \emph{Authors (Group X):}\\
            \theauthor
            \end{flushleft}
            \end{minipage}~
            \begin{minipage}{0.4\textwidth}
            \begin{flushright} \large
            \emph{Student Number:} \\
            260612056 \\ 260612056 \\ 260612056                                  % Your Student Number
        \end{flushright}
    \end{minipage}\\[2 cm]
 
    {\large \thedate}\\[2 cm]
 
    \vfill
    
\end{titlepage}

%%%%%%%%%%%%%%%%%%%%%%%%%%%%%%%%%%%%%%%%%%%%%%%%%%%%%%%%%%%%%%%%%%%%%%%%%%%%%%%%%%%%%%%%%

\section*{2.1 Preparation}

\subsection*{2.1.2 Considering this circuit will be used as an output stage in an operational amplifier, what are the main characteristics that it must have?}

INSERT ANSWER HERE

\subsection*{2.1.3 What is the purpose of $R_2$ and $R_3$?}

INSERT ANSWER HERE

\subsection*{2.1.4 Using $V_{EE} = -5V$ and $V_{CC} = 5V^1$, design the current source of the differential pair so that it sinks a current of $0.25mA$. Determine the required value of $R$.}

INSERT ANSWER HERE

\subsection*{2.1.5 Considering the current source only, plot the source’s output current versus the output voltage and comment on the results.}

\begin{figure}[H]
  \centering
    \subfloat[]{{\includegraphics[width=0.4\textwidth]{image1.jpg} }}%
    \qquad
    \subfloat[]
    {{\includegraphics[width=0.4\textwidth]{image2.jpg} }}%
    \qquad
    \caption{sample image with caption}%      
    \label{fig:testing}%
\end{figure}

\subsection*{2.1.6 Design the circuit of Figure 2.1 in order to obtain a gain of $250V/V ±10\%$. Determine the values of $R_4$ and $R_5$. Discuss. At the output, take into account the loading caused by a $10X$ oscilloscope probe.}

Note \\
The P2220 oscilloscope probe can be used in 1X mode or 10X mode. The equivalent circuit models of this probe (including the oscilloscope) for each mode are quite different:  \\
- 1X mode: 1 MΩ in parallel with a 110 pF capacitance to ground.  \\
- 10X mode: 10 MΩ in parallel with a 17 pF capacitance to ground.  \\ Usually, the probe should be used in the 10X mode if possible to reduce loading effects (both capacitive and resistive). Note that a 10X probe attenuates signals by 10 times, so it may not be optimal to use the probe in this mode to measure very small signals.   Make sure to compensate your probe (see oscilloscope/probe manual) before taking measurements in order to ensure that your measured voltages are valid! If you use the oscilloscope directly with no probe (through a RG-58 coaxial cable), you load on a circuit will be similar to that of the probe in 1X mode. More specifically, the load resistance will be of 1 MΩ, and the parallel capacitance will be estimated by adding the 20 pF load of the oscilloscope to ~80 pF per meter of cable used between the circuit and oscilloscope (e.g., total capacitive load of ~100 pF for a 1 m cable)

\subsection*{2.1.7 Plot the voltage transfer characteristics and accompany it with the corresponding time domain waveform plot. Discuss the curve and maximal output swing.}

INSERT ANSWER HERE

\subsection*{2.1.8 Determine, if any, the input DC offset required to maximize the output swing. Document the maximum output swing. Discuss.}

INSERT ANSWER HERE

\subsection*{2.1.9 Plot the differential mode frequency response and determine the $3-dB$ points. Also, take into account the parasitic resistor and capacitor values of the $10X$ oscilloscope probe at the output. Discuss.  }

INSERT ANSWER HERE

\subsection*{2.1.10 Plot the common mode frequency response. Also, take into account the parasitic resistor and capacitor values of the $10X$ oscilloscope probe at the output. Discuss. }

\subsection*{2.1.11 Find the input and output resistances the circuit. Discuss. }

INSERT ANSWER HERE

\subsection*{2.1.12 In summary, what are the advantages and limitations of this circuit setup?}

INSERT ANSWER HERE


\end{document}