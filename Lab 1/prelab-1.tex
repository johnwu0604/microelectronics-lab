\documentclass[12pt]{article}
\usepackage[english]{babel}
\usepackage{natbib}
\usepackage{url}
\usepackage[utf8x]{inputenc}
\usepackage{amsmath}
\usepackage{graphicx}
\graphicspath{{images/}}
\usepackage{parskip}
\usepackage{fancyhdr}
\usepackage{vmargin}
\usepackage{subfig}
\usepackage{float}
\setmarginsrb{3 cm}{2.5 cm}{3 cm}{2.5 cm}{1 cm}{1.5 cm}{1 cm}{1.5 cm}

\title{2.1 - Differential Amplifier (Prelab)}                             % Title
\author{John Wu \\ Ian Smith \\ Bilal Yousuf}                               % Author
\date{\today}                                                           % Date

\makeatletter
\let\thetitle\@title
\let\theauthor\@author
\let\thedate\@date
\makeatother

\pagestyle{fancy}
\fancyhf{}
\rhead{\theauthor}
\lhead{\thetitle}
\cfoot{\thepage}

\begin{document}

%%%%%%%%%%%%%%%%%%%%%%%%%%%%%%%%%%%%%%%%%%%%%%%%%%%%%%%%%%%%%%%%%%%%%%%%%%%%%%%%%%%%%%%%%

\begin{titlepage}
    \centering
    \vspace*{0.5 cm}
    \includegraphics[scale = 0.07]{mcgill-logo.png}\\[1.0 cm]   % University Logo
    \textsc{\LARGE McGill University}\\[1.0 cm]   % University Name
    \textsc{\Large ECSE 335}\\[0.5 cm]               % Course Code
    \textsc{\large Microelectronic Labs}\\[0.5 cm]               % Course Name
    \rule{\linewidth}{0.2 mm} \\[0.4 cm]
    { \huge \bfseries \thetitle}\\
    \rule{\linewidth}{0.2 mm} \\[1.5 cm]
    \begin{minipage}{0.4\textwidth}
        \begin{flushleft} \large
            \emph{Authors (Group X):}\\
            \theauthor
            \end{flushleft}
            \end{minipage}~
            \begin{minipage}{0.4\textwidth}
            \begin{flushright} \large
            \emph{Student Number:} \\
            260612056 \\ 260612056 \\ 260680182                                  % Your Student Number
        \end{flushright}
    \end{minipage}\\[2 cm]
 
    {\large \thedate}\\[2 cm]
 
    \vfill
    
\end{titlepage}

%%%%%%%%%%%%%%%%%%%%%%%%%%%%%%%%%%%%%%%%%%%%%%%%%%%%%%%%%%%%%%%%%%%%%%%%%%%%%%%%%%%%%%%%%

\section*{2.1 Preparation}

\begin{figure}[H]
\centering
\includegraphics[width=0.6\textwidth]{op_amp.PNG}
\caption{\label{fig:op-amp} Differential amplifier with active load}
\end{figure}

\subsection*{2.1.2 Considering this circuit will be used as an output stage in an operational amplifier, what are the main characteristics that it must have?}

Since the circuit will be used as an output stage in an operational amplifier, it must have the following characteristics in its most ideal form.


\begin{itemize}
    \item Infinite open-loop gain.
    \item Infinite bandwidth due to ideal gain.
    \item Infinite or zero common mode rejection ratio (CMRR).
    \item Infinite input impedance.
    \item Zero output impedance.
\end{itemize}

In general, these ideal characteristics will not be the actual case, but it will aim to be as close to these idealistic values as possible in order to deliver large currents to a small load - the goal of an op-amp.

\subsection*{2.1.3 What is the purpose of $R_2$ and $R_3$?}

The resistors, $R_2$ and $R_3$, are source degeneration resistances used for creating a higher output impedance and a lower equivalent transconductance. These resistors are used to give a high resistance to the current mirror formed by $Q_5$ and $Q_6$. The high output resistance will reduce the common mode gain and improve the CMRR. The resistors prevent thermal runaway in transistors Q5 and Q6 thereby providing thermal stability.

\subsection*{2.1.4 Using $V_{EE} = -5V$ and $V_{CC} = 5V^1$, design the current source of the differential pair so that it sinks a current of $0.25mA$. Determine the required value of $R$.}

When $R_2 = R_3 = 2k\Omega$ and $Q_5$ and $Q_6$ make perfect current mirrors: \\

$I = \frac{V_{EE} - V_{BE} - V_{CE}}{R_2 + R_1}$ \\

$\frac{9.3V}{2k\Omega + R_1} = 0.25mA$ \\

$R_1 = 35.2k\Omega$


\subsection*{2.1.5 Considering the current source only, plot the source’s output current versus the output voltage and comment on the results.}

\begin{figure}[H]
\centering
\includegraphics[width=0.9\textwidth]{Screenshots/Circuit_215.png}
\caption{\label{fig:current-circuit} Circuit used for analyzing current source.}
\end{figure}

\begin{figure}[H]
\centering
\includegraphics[width=0.9\textwidth]{Screenshots/Graph_215.png}
\caption{\label{fig:current-voltage} Output current versus output voltage graph.}
\end{figure}

It can be observed that the output current reaches the expected constant value of approximately $0.25mA$ regardless of the output voltage.

\subsection*{2.1.6 Design the circuit of Figure 2.1 in order to obtain a gain of $250V/V ±10\%$. Determine the values of $R_4$ and $R_5$. Discuss. At the output, take into account the loading caused by a $10X$ oscilloscope probe.}

$r_e = \frac{V_T}{I_C/2} = \frac{2*25mV}{0.25mA} = 200\Omega$ \\
\\$r_{o2} = \frac{V_A}{{I_C}/2} = \frac{2*115.7V}{0.25mA} = 925.6k\Omega$\\
\\$r_{o4} = \frac{V_A}{I_C/2} = \frac{2*90.7V}{0.25mA} = 725.6k\Omega$\\
\\$g_m = \frac{I_C}{V_T} = \frac{0.125mA}{25mV} = 5mA/V$\\
\\$A_d = \frac{r_{o4}(1+g_mR_4)||r_{o2}||R_{OSC}}{r_e + R_4}$\\
\\We consider $R_{OSC}$ = 10 M\Omega$.\\
\\$R_4$ is approximately $1.36k\Omega$. We'll make $R_5 = R_4 = 1.36k\Omega$\\

\subsection*{2.1.7 Plot the voltage transfer characteristics and accompany it with the corresponding time domain waveform plot. Discuss the curve and maximal output swing.}

\begin{figure}[H]
\centering
\includegraphics[width=0.9\textwidth]{Screenshots/Graph_217_Vcharacteristic.png}
\caption{\label{fig:voltage-transfer} Voltage transfer characteristics.}
\end{figure}

\begin{figure}[H]
\centering
\includegraphics[width=0.9\textwidth]{Screenshots/Graph_217_time.png}
\caption{\label{fig:time-domain} Time domain waveform.}
\end{figure}

We can observe that maximal output swing centers at around 2V and occurs at input signals between -50mV and 50mV.

\subsection*{2.1.8 Determine, if any, the input DC offset required to maximize the output swing. Document the maximum output swing. Discuss.}

The max DC offset that results in the maximum output voltage swing occurs at the centre of the steepest part of the curve. As mentioned above, this offset value is around 2V.

\subsection*{2.1.9 Plot the differential mode frequency response and determine the $3-dB$ points. Also, take into account the parasitic resistor and capacitor values of the $10X$ oscilloscope probe at the output. Discuss.  }

\begin{figure}[H]
\centering
\includegraphics[width=0.9\textwidth]{Screenshots/Graph_217_time.png}
\caption{\label{fig:time-domain} Time domain waveform.}
\end{figure}

We can observe above that the differential gain is X and the 3 dB cutoff frequency is Y.

\subsection*{2.1.10 Plot the common mode frequency response. Also, take into account the parasitic resistor and capacitor values of the $10X$ oscilloscope probe at the output. Discuss. }

\begin{figure}[H]
\centering
\includegraphics[width=0.9\textwidth]{Screenshots/Graph_217_time.png}
\caption{\label{fig:time-domain} Time domain waveform.}
\end{figure}

The common mode gain is X, which is significantly lower than our differential gain, and therefore meets the requirements of the circuit.

\subsection*{2.1.11 Find the input and output resistances the circuit. Discuss. }

$R_{in} = 2 (\beta + 1) (r_e + R_4) = 2 (100 + 1) (200+1.36k) = 315.3k \Omega $ \\
$R_{out} = r_{o2} || r_{o4} (1 + g_m R_5 || r_{\pi}) = 925.6k || 725.6k ( 1 + 0.005 (1.36k || 20k ) ) = 795.4k \Omega$

For $R_{out}$ the resistance of the 10X probe was not considered in the calculation since it is large enough that we can neglect it.$

\subsection*{2.1.12 In summary, what are the advantages and limitations of this circuit setup?}

The advantages of this circuit are that is gives a high gain with a flexible voltage swing. The limitations are that there is a large output resistance with a narrow bandwidth.

\end{document}